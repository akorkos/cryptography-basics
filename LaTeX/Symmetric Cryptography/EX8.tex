\section{Άσκηση 4.3}

\subsection{Εκφώνηση}

Αν Σ ένα σύνολο με |Σ| συμβολίζουμε το πλήϑος των στοιχείων του, υπολογίστε τη διαφορική ομοιομορφία (differential uniformity) του S-box,

\begin{equation}
     Diff(S) = \max_{x \in \{0,1\}^6 - \{0\}, y\in \{0,1\}^4} \{|z \in \{0,1\}^6 : S(z \oplus x) \oplus S(z) = y\}|
\end{equation}

Γενικά, για S-boxes:

\begin{equation}
    S : \{0, 1\}^n \rightarrow \{0, 1\}^m
\end{equation}

ο προηγούμενος ορισμός γράφεται:

\begin{equation}
     Diff(S) = \max_{x \in \{0,1\}^n - \{0\}, y\in \{0,1\}^m} \{|z \in \{0,1\}^n : S(z \oplus x) \oplus S(z) = y\}|
\end{equation}

και ισχύει:

\begin{equation}
    Diff(S) \ge \max \{2, 2^{n-m}\}
\end{equation}

Όσο μικρότερη είναι αυτή η ποσότητα, τόσο πιο ανθεκτικό είναι το S-box στη διαφορική κρυπτανάλυση.