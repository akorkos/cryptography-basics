\section{Άσκηση 2.2}

\subsection{Εκφώνηση} 

Το επόμενο κρυπτόγραμμα έχει ληφθεί:

\begin{center}
    \textbf{οκηθμφδζθγοθχυκχσφθμφμχγ}
\end{center}
Ο αλγόριθμος κρυπτογράφησης είναι ο εξής: 

Κάθε γράμμα του αρχικού μας μηνύματος αντικαθίσταται από την αριθμητική του τιμή (α: 1, ..., ω: 24). Ας είναι $x_0$ μία ρίζα του τριωνύμου $g(x) = x^2 + 3x + 1$. Σε κάθε αριθμό του μηνύματός μου προσθέτω την τιμή του πολυωνύμου $f(x) = x^5 + 3x^4 + 3x^3 + 7x^2 + 5x + 4$, στο $x_0$. Αντικαθιστώ κάθε αριθμό με το αντίστοιχο γράμμα. Βρείτε το αρχικό μήνυμα.

\subsection{Λύση} 

Από την εκφώνηση, γνωρίζω ότι $g(x) = 0$. 

Επίσης, θα απλοποιήσω την $f(x)$ μέσω της διαίρεση πολυωνυμικών συναρτήσεων έτσι ώστε να εκμεταλλευτώ την παραπάνω πληροφορία. Συνεπώς: 

\begin{equation} \label{eq:121}
    f(x) = g(x) \cdot h(x) \Rightarrow h(x) = \frac{f(x)}{g(x)} \Rightarrow h(x) = x^3+2x+1+\frac{3}{x^2+3x+1}
\end{equation}

Κάνοντας εφαρμογή της επιμεριστικής ιδιότητας στην (\ref{eq:121}), προκύπτει: 

\begin{equation} \label{eq:122}
    f(x) = (x^3+2x+1)\cdot g(x) + \frac{3}{\cancel{x^2+3x+1}}\cdot \cancel{g(x)} \Rightarrow f(x) = (x^3+2x+1) \cdot g(x) + 3
\end{equation}

Εξετάζω την συνάρτηση (\ref{eq:122}) για $x = x_0$:

\begin{equation} \label{eq:123}
    f(x_0) = (x_{0}^3+2x_0+1) \cdot \cancelto{0}{g(x_0)} + 3 \Rightarrow f(x_0) = 3
\end{equation}

