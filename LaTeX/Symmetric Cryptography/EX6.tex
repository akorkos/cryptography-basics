\section{Άσκηση 3.6}

\subsection{Εκφώνηση}

Αποδεικνύεται ότι το πλήθος των ανάγωγων πολυωνύμων βαθμού n στο σώμα $\mathbb{F}_2$ είναι

\begin{equation}
    N_2(n) = \frac{1}{n} \sum_{d|n}\mu(d) \cdot 2^{n/d},
\end{equation}

όπου

\begin{equation}
    \mu(d) = 
    \begin{cases}
      1 & d = 1 \\
      (-1)^k & d = p_1p_2\cdot\cdot\cdotp_k p_i :\text{πρώτοι}\\
      0 & \text{αλλού}
    \end{cases}
\end{equation}

Με το σύμβολο $d|n$ εννοούμε όλους τους θετικούς διαιρέτες του n. π.χ. αν $n = 30$, τότε

\begin{equation}
    \{d|n: 1 \le d \le n\} = \{1, 2, 3, 5, 6, 10, 15, 30\}
\end{equation}

Με χρήση του συστήματος sagemath υπολογίστε το $N_2(10)$.